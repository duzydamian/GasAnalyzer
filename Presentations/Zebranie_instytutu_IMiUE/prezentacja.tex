\documentclass[ucs]{beamer}
\usepackage{polski}
\usepackage[utf8]{inputenc}
\usepackage{pgfplots}
\usepackage{tikz}
\usetikzlibrary{automata,positioning,shapes,shadows,arrows,backgrounds,trees,fit,calc,decorations.pathreplacing,decorations.markings}

\usetheme{Ilmenau}
%\usetheme{Marburg}

\title[Współpraca międzywydziałowa]{Współpraca międzywydziałowa pomiędzy Wydziałem Automatyki, Elektroniki i Informatyki, a Wydziałem Inżynierii Środowiska i Energetyki}
\subtitle{Realizacja aplikacji Gas Analyzer z wykorzystaniem protokołu ELAN}
\author[Karbowiak, Powała]{mgr inż. Damian Karbowiak \inst{1} \and mgr inż. Grzegorz Powała \inst{2}}
\institute[Politechnika Śląska]{\inst{1} Zespół Urządzeń Informatyki \and %
                      \inst{2} Zespół Mikroinformatyki i Teorii Automatów Cyfrowych %
                      \\\vspace{2mm} \includegraphics[height=0.2\textheight]{images/PolslLogo} }
\date{\today}
       
\setbeamertemplate{footline}[frame number]
\useoutertheme{infolines}
 
\begin{document}

\begin{frame}
  \titlepage
\end{frame}


\section{Wstęp}
\begin{frame}
\frametitle{Historia}
\begin{itemize}
\item 22 luty 2013 \\
Pierwszy kontakt mailowy z Panem Kress
\item 28 luty 2013 \\
Pierwsze spotkanie w sprawie współpracy
\item 21 marzec 2013 \\
Wypożyczenie Ultramatu 23 i rozpoczęcie współpracy oraz realizacji projektu
\item kwiecień -- czerwiec 2013 \\
Realizacja projektu
\item wrzesień 2013 \\
Finalizacja pierwszej części i podstawowej wersji projektu projektu
\end{itemize}
\end{frame}

\begin{frame}
\frametitle{Współpraca}
\begin{center}
\begin{tabular}{cccccc}
\includegraphics[width=0.1\textwidth]{images/AEiILogo} &
\includegraphics[width=0.1\textwidth]{images/IILogo} &
\includegraphics[width=0.25\textwidth]{images/SKNIndustrumLogo} &
\includegraphics[width=0.1\textwidth]{images/WISiELogo} &
\includegraphics[width=0.1\textwidth]{images/IMIUELogo} &
\includegraphics[width=0.1\textwidth]{images/ZKiWPLogo} \\ 
\end{tabular} 
\end{center}

\begin{enumerate}
\item Wydział Automatyki, Elektroniki i Informatyki
\begin{itemize}
\item Instytut Informatyki
\begin{itemize}
\item Koło Naukowe Przemysłowych Zastosowań Informatyki ,,Industrum''
\\ mgr inż. Damian Karbowiak
\\ mgr inż. Grzegorz Powała
\end{itemize}
\end{itemize}
\item  Wydział Inżynierii Środowiska i Energetyki
\begin{itemize}
\item Instytut Maszyn i Urządzeń Energetycznych
\begin{itemize}
\item Zakład Kotłów i Wytwornic Pary
\\ mgr inż. Tomasz Kress
\end{itemize}
\end{itemize}
\end{enumerate}
\end{frame}


\section{Współpraca}
\begin{frame}
\frametitle{Możliwości}
\begin{enumerate}
\item Instytut Informatyki
\begin{itemize}
\item Wiedza informatyczna
\item Specjalizacja związana ze stosowaniem informatyki w przemyśle
\item Koło naukowe o tematyce przemysłowej
\item Projekty zaliczeniowe semestralne oraz prace inżynierskie i magisterskie
\item Studenci chętni do realizacji projektów praktycznych z wykorzystaniem istniejącego sprzętu i stanowisk laboratoryjnych
\end{itemize} 
\item Instytut Maszyn i Urządzeń Energetycznych
\begin{itemize}
\item Potrzeba informatyzacji
\item Ciekawe problemy informatyczne
\item Spora ilość sprzętu i stanowisk
\item Ciekawe pomysły i potrzeby na oprogramowanie/sprzęt
\end{itemize}
\end{enumerate}
\end{frame}

\begin{frame}
\frametitle{Gas Analyzer - geneza}
\begin{enumerate}
\item Realizacja pomiarów przemysłowych
\item Wykorzystywanie kilku analizatorów firmy SIEMENS
\item Zapisywanie pomiarów w tabelce na kartce
\item Ograniczona częstotliwość pomiarów
\end{enumerate}
\end{frame}

\begin{frame}
\frametitle{Gas Analyzer - realizacja}
\begin{enumerate}
\item Wykorzystanie protokołu komunikacyjnego ELAN
\item Możliwość podłączenia do 12 analizatorów firmy SIEMENS:
\begin{itemize}
\item ULTRAMAT 6
\item OXYMAT 6 / OXYMAT 61
\item CALOMAT 6
\item ULTRAMAT 23
\end{itemize}
\item Automatyczny odczyt stanu urządzeń
\item Możliwość archiwizacji pomiarów z dowolnym interwałem czasowym, z~rozdzielczością co sekundę
\item Automatyczne wykrywanie urządzeń i wielkości mierzonych
\end{enumerate}
\end{frame}

\begin{frame}
\frametitle{ELAN -- Podłączenie}
\begin{figure}[htbp]
 \centering
        \tikzstyle{background grid}=[draw, black!50,step=.25cm]
	\begin{tikzpicture}[scale=0.2]%, show background grid]
	\tikzset{
    	mynode/.style={rectangle,rounded corners,draw=black, top color=white, bottom color=yellow!50,very thick, inner sep=1em, minimum size=3em, 		text centered, text width=3.3cm},
    	mynodemini/.style={rectangle,rounded corners,draw=black, top color=white, bottom color=yellow!50,very thick, inner sep=.5em, text centered},    	
	    myarrow/.style={->, >=latex', shorten >=1pt, thick},
	    myline/.style={-, =latex', shorten >=1pt, rounded corners, ultra thick},
	    mylabel/.style={text width=7em, text centered} 
	} 
	\node[mynode] (plc) {Komputer \\ przemysłowy C6925};  
	\node [left=of plc] (laptop) {\includegraphics[width=3cm]{images/laptop}};
	\node[below=2cm of plc] (dummy) {}; 
	\node[mynode, left=of dummy] (io) {Wyspa EK1100 z~zestawem modułów IO};  	 	 	
 	\node[mynodemini, left=of io] (io2) {EL1004};
 	\node[mynodemini, above=2mm of io2] (io1) {EL4132};  	
 	\node[mynodemini, above=2mm of io1] (io0) {EK1100};
 	\node[mynodemini, below=2mm of io2] (io3) {EL2004}; 	 	
 	\node[mynodemini, below=2mm of io3] (io4) {EL2004};
 	\node[mynodemini, below=2mm of io4] (io5) {EL3102};
 	\node [fit=(io0) (io1) (io2) (io3) (io4) (io5)] (fit) {};  
 	%\draw [decorate, xshift=-20pt,line width=4pt] (fit.south east) -- (fit.north east);
	\draw [decorate,decoration={brace, mirror,amplitude=10pt}, line width=1pt] (fit.south east) -- (fit.north east);
	
	\node[mynode, right=of dummy] (naped) {Napęd\\serwomechnizmów AX5203};
	\node[below=2cm of naped] (dummy2) {}; 
	\node[mynode, left=of dummy2, text width=4cm](silnik1){Silnik \\ AM3021-0C00-0000};
	\node[mynode, right=of dummy2, text width=4cm](silnik2){Silnik \\ AM3021-0C40-0000};
	
	\draw[myline,black,dotted] (fit.east) ++(0.4, 0) -- (io.west);
	\draw[myline,blue] (laptop.east) -- ++(-1, 0) -- (plc.west);
	
	\draw[myline,purple] (io0.south) -- (io1.north);	
	\draw[myline,purple] (io1.south) -- (io2.north);
	\draw[myline,purple] (io2.south) -- (io3.north);		
	\draw[myline,purple] (io3.south) -- (io4.north);
	\draw[myline,purple] (io4.south) -- (io5.north);
	
	\draw[myline,yellow] (plc.south) -- (naped.north);	
	\draw[myline,yellow] (io.east) -- (naped.west);	

	\draw[myline, green, bend right=10] (naped.south) to (silnik1.north);
	\draw[myline, orange, bend left=10] (naped.south) to (silnik1.north);	
	\draw[myline, green, bend right=10] (naped.south) to (silnik2.north);
	\draw[myline, orange, bend left=10] (naped.south) to (silnik2.north);	
	%\draw[<->, >=latex', shorten >=2pt, shorten <=2pt, bend right=45, thick, dashed] 
    %(io.south) to node[auto, swap] {Competition}(naped.south); 
    
	\draw [fill=blue!5, thick] (5.75,1.5) rectangle (9,-3);

    \draw [purple, line width=6] (6,1) -- (6.5,1); \node[text width=2cm] at (7.65,0.7) {EtherCAT (E-bus)};    
    \draw [yellow, line width=6] (6,0) -- (6.5,0); \node[text width=2cm] at (7.65,-0.3) {EtherCAT (skrętka)};
    \draw [blue, line width=6] (6,-1) -- (6.5,-1); \node at (7.5,-1) {Ethernet};
    \draw [black, line width=6] (6,-1.5) -- (6.5,-1.5); \node at (7.5,-1.5) {Zasilanie};
    \draw [green, line width=6] (6,-2) -- (6.25,-2); \draw [orange, line width=6] (6.25,-2) -- (6.5,-2); 
    \node [text width=2cm] at (7.75,-2.25) {Sterowanie silnikiem};    
\end{tikzpicture} 
\caption{Schemat stanowiska typu CP.}
\label{stanowisko:cp}
\end{figure}
\end{frame}

\section{Podsumowanie}
\begin{frame}
\frametitle{Wnioski}
\begin{itemize}
\item Liczne perspektywy współpracy
\end{itemize}
\end{frame}

\begin{frame}
\frametitle{Podsumowanie}
Dziękujemy za uwagę.
\end{frame}

\begin{frame}
\frametitle{Pytania}
Czas na pytania.
\end{frame}

\end{document}