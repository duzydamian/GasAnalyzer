\tikzstyle{background grid}=[draw, black!50,step=.25cm]
\newcounter{wavenum}
\newcounter{iter}

\setlength{\unitlength}{1cm}

% advance clock one cycle, not to be called directly
\newcommand*{\clki}[1]{
  \draw[white] (t_cur) -- ++(#1,0) node[time] (t_cur) {};
}

\newcommand*{\bitvector}[3]{
  \draw[fill=#3] (t_cur) -- ++( .1, .3) -- ++(#2-.2,0) -- ++(.1, -.3)
                         -- ++(-.1,-.3) -- ++(.2-#2,0) -- cycle;
  \path (t_cur) -- node[anchor=mid] {#1} ++(#2,0) node[time] (t_cur) {};
}

% \known{val}{length}
\newcommand*{\known}[2][]{
    \bitvector{#1}{#2}{white}
}

% \unknown{length}
\newcommand*{\unknown}[2][]{
    \bitvector{#1}{#2}{black!30}
}

% \bit{1 or 0}{length}
\newcommand*{\bit}[2]{
  \draw (t_cur) -- ++(0,.6*#1-.3) -- ++(#2,0) -- ++(0,.3-.6*#1)
    node[time] (t_cur) {};
}
\newcommand*{\lbl}[2]{
 \path (t_cur) -- node[time] {} ++(#2-.2,0) node[anchor=mid] (t_cur) {#1} ++(.2,0) node[time] (t_cur) {};
}

% \nextwave{name}
\newcommand{\nextwave}[1]{
  \path (0,\value{wavenum}) node[left=0.5cm] {#1} node[time] (t_cur) {};
  \addtocounter{wavenum}{-1}
}

\newcommand{\nr}{
	$\Rightarrow$
}

\newcommand{\ur}{
	$\circlearrowleft$
}

% \clk{name}{period}
\newcommand{\clk}[2]{
    \nextwave{#1}
    \FPeval{\res}{(\wavewidth+1)/#2}
    \FPeval{\reshalf}{#2/2}
    \foreach \t in {1,2,...,\res}{
        \bit{\reshalf}{1}
        \bit{\reshalf}{0}
    }
}
    
% \begin{wave}[clkname]{num_waves}{clock_cycles}
\newenvironment{wave}[3][]{
  \begin{tikzpicture}[draw=black, yscale=.7,xscale=1]%, show background grid]
    \tikzstyle{time}=[coordinate]
    \setlength{\unitlength}{1cm}
    \def\wavewidth{#3}
    \setcounter{wavenum}{0}
    \nextwave{#1}
    \def\myarray{1,5,1}
    %\foreach \t in {0,1,...,\wavewidth}{
    \foreach \i / \x in {0/1.5, 1/2, 2/1.25, 3/2.5, 4/.75, 5/1, 6/1.75} {
      \draw[dotted] (t_cur) +(0,.1) node[above] {$t_\i$} -- ++(0,.4-#2);
      \clki{\x};
    }
    \draw[->, >=latex', shorten >=1pt, thick] (-.4,-3.5) -- (-.4,0);
    \draw[->, >=latex', shorten >=1pt, thick] (-.4,-3.5) -- (10,-3.5);
}
{\end{tikzpicture}}

\begin{wave}{4}{15}
 \nextwave{Bufor 1}  \lbl{\nr}{0} \known{.5} \lbl{}{1} \unknown{.5} \lbl{\ur}{1.5} \unknown{.5} \lbl{}{.75} \unknown{.5} \lbl{}{2} \unknown{.5} \lbl{}{1.25} \known{.5}
 \nextwave{Bufor 2}  \known{.5} \lbl{}{1} \known{.5} \lbl{}{1.5} \known{.5} \lbl{\nr}{.75} \known{.5} \lbl{}{2} \unknown{.5} \lbl{\nr}{1.25} \known{.5}
 \nextwave{Bufor 3} \known{.5} \lbl{\nr}{1} \known{.5}	\lbl{}{1.5} \unknown{.5} \lbl{}{.75} \unknown{.5} \lbl{\ur}{2} \unknown{.5} \lbl{}{1.25} \known{.5}
\end{wave}

$t_0$ -- nadejście pomiaru z urządzenia 1 \\
$t_1$ -- nadejście pomiaru z urządzenia 3 \\
$t_2$ -- nadejście pomiaru z urządzenia 1 \\
$t_3$ -- nadejście pomiaru z urządzenia 2 \\
$t_4$ -- nadejście pomiaru z urządzenia 3 \\
$t_5$ -- Migawka, czyli zapis wszystkich buforów do bazy \\
$t_6$ -- nadejście pomiaru z urządzenia 2 