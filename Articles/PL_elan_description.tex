\section{Opis protokołu ELAN}

ELAN (ang. Economical Local Area Network), czyli ekonomiczna sieć lokalna został wprowadzony przez firmę SIEMENS w swoich analizatorach składu gazu. Protokół ten według twórców został wprowadzony jako ekonomiczny interfejs szeregowy do transmisji wartości mierzonych pomiędzy analizatorami oraz prostej komunikacji z komputerami PC dla celów testowych i serwisowych. Najważniejsze parametry zebrano w Tablicy~\ref{tab:parametry}.
Zostało wprowadzone ograniczenie maksymalnej liczby urządzeń pracujących w sieci do 14 (2 urządzenia kontrolne/komputery oraz do 12 analizatorów). Obsługiwane analizatory firmy SIEMENS:
\begin{itemize}
\item ULTRAMAT 6
\item OXYMAT 6 / OXYMAT 61
\item CALOMAT 6
\item ULTRAMAT 23
\end{itemize}
Komunikacja została oparta na następujących założeniach:
\begin{itemize}
\item Wszystkie podpięte analizatory mają te same prawa
\item Aby uniknąć konfliktów, każdy analizator musi sprawdzić stan magistrali i zatrzymać transmisję natychmiast w razie potrzeby (mechanizm CSMA / CD)
\item Nowa komenda może zostać wysłana dopiero gdy poprzednia komenda zostanie potwierdzona (z wyjątkiem trybu rozgłoszeniowego)
\end{itemize}

\begin{table}[h]
\centering
\begin{tabular}{|l|l|}
\hline Poziom & RS485 \\ 
\hline Szybkość transmisji & 9600 \\ 
\hline Bity danych & 8 \\ 
\hline Bit startu & 1 \\ 
\hline Bit stopu & 1 \\ 
\hline Kontrola parzystości & nie \\ 
\hline no ECHO &  \\ 
\hline 
\end{tabular} 
\caption{Parametry interfejsu}
\label{tab:parametry}
\end{table}

%\subsection{Analiza statystyczna protokołu}
%TCS -- Rozmiar transmitowanego znaku (ang. transmitted char size)\\
%CS -- rozmiar znaku (ang. char size)\\
%SpB -- bit stopu (ang. Stop bit) \\
%StB -- bit startu (ang. Start bit)
%$$TCS = CS + StB + SpB= 8+1+1=10 [b]$$
%CPS -- liczba znaków na sekundę (ang. chars per second) \\
%CBR -- znakowa szybkość transmisji (ang. char baud rate) \\
%BR -- szybkość transmisji (ang. baud rate) \\
%Szybkość transmisji wg dokumentacji: 9600 bps (bitów na sekundę)
%$$CBR=BR/TCS=9600/10=960 [cps]$$

\subsection{Ramka}
Każda ramka w sieci  wygląda ogólnie jak na Rysunku~\ref{elan:ramka}. W pierwszej wersji, aktualnie zrealizowanej przez autorów obsługiwane są jedynie ramki rozgłoszeniowe.
\begin{figure}[htbp]
 \centering
        \tikzstyle{background grid}=[draw, black!50,step=.25cm]
	\begin{tikzpicture}[node distance=1.5mm, auto]%, show background grid]
	\tikzset{
    	mynode/.style={rectangle,rounded corners,draw=black, top color=white, very thick, inner sep=4mm, 		text centered,font=\footnotesize},
    	mynodemini/.style={rectangle,rounded corners,draw=black, top color=white, thick, inner sep=2mm, text centered,font=\scriptsize},    	
	    myarrow/.style={->, >=latex', shorten >=1pt, ultra thick},
	    myline/.style={-, =latex', shorten >=1pt, rounded corners, ultra thick},
	    mylabel/.style={text centered, font=\scriptsize\bfseries} 
	} 
	\node[bottom color=gray!50, mynode, text width=2.3cm] (start) {Start};  
	\node[bottom color=gray!50, mynodemini, below=of start.213, text width=1cm] (dle) {DLE (10H)};	
	\node[bottom color=gray!50, mynodemini, right=of dle, text width=1cm] (soh) {SOH (01H)};  
	
	\node[bottom color=yellow!50, mynode, right=of start, text width=6.3cm] (used_data) {USED DATA};
	\node[bottom color=yellow!50, mynodemini, below=of used_data.190] (ud1) {TA};  	 
	\node[bottom color=yellow!50, mynodemini, right=of ud1] (ud2) {SA};
	\node[bottom color=yellow!50, mynodemini, right=of ud2] (ud3) {CCS};
	\node[bottom color=yellow!50, mynodemini, right=of ud3] (ud4) {CS};
	\node[bottom color=yellow!50, mynodemini, right=of ud4] (ud5) {Komenda};
	\node[bottom color=yellow!50, mynodemini, right=of ud5] (ud6) {Dane};			

	\node [fit=(ud1) (ud2) (ud3) (ud4) (ud5) (ud6)] (fit) {};  
 	%\draw [decorate, xshift=-20pt,line width=4pt] (fit.south east) -- (fit.north east);
	\draw [decorate,decoration={brace,amplitude=10pt}, line width=1pt] (fit.south east) -- (fit.south west);		
	\node[mylabel, below=4mm of fit] (fits) {max. 68 characters, 10H zawsze podwojone};
			
	\node[bottom color=gray!50, mynode, right=of used_data, text width=2.2cm] (eot) {End of \\ transmission};
	\node[bottom color=gray!50, mynodemini, below=of eot.222, text width=1cm] (dle2) {DLE (10H)};
	\node[bottom color=gray!50, mynodemini, right=of dle2, text width=1cm] (etx) {ETX (03H)}; 	 
	
	\node[bottom color=gray!50, mynode, right=of eot] (crc16) {CRC16};
	
	\node[mylabel, below=2cm of soh] (dal) {TA -- Target Address};
	\node[mylabel, right=of dal] (sal) {SA -- Source Address};
	\node[mylabel, right=of sal] (padl) {Pad. -- Payload};
	\node[mylabel, right=of padl, text width=4cm] (fcsl) {FCS -- Frame Check Sequance (CRC)};			
	
\end{tikzpicture} 
\caption{Ramka w protokole ELAN.}
\label{elan:ramka}
\end{figure} %

\subsection{Topologia}
Jak już zostało to opisane wcześniej w sieci może pracować do 12 analizatorów oraz komputer co zostało przedstawione na Rysunku~\ref{elan:topologia}. Urządzenia pomiarowe podłączone są, w przypadku zrealizowanego stanowiska, do konwertera RS485 $\Leftrightarrow$ USB.
\input{tikz/topology} %

\subsection{Tryb rozgłoszeniowy}
W aktualnej wersji projektu twórcy wykorzystują dostępny standardowo w protokole tryb rozgloszeniowy (ang. broadcast). W trybie tym każdy analizator automatycznie i~cyklicznie, co 500 ms, transmituje wszystkie swoje aktualne wartości zmierzone do pozostałych urządzeń.