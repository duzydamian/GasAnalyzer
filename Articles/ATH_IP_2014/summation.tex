\section{Podsumowanie}
Stworzona aplikacja oparta, o zaproponowane rozwiązanie, pozwoliła znacząco uprościć i przyśpieszyć proces pomiarowy.Rozwiązanie to jest bardzo tanie i proste w uruchomieniu. Stworzone oprogramowanie pozwala na swobodne modyfikowanie listy urządzeń, które są automatycznie wykrywane. \vspace{1cm} \\
EWENTUALNIE!!!!!!! z naszego raportu \\
Nadrzędnym celem projektu było stworzenie oprogramowania gromadzącego i zapisującego dane z analizatorów spalin firmy Siemens. Urządzenia te są wyposażone w dwa interfejsy komunikacyjne, w tym sugerowany przez producenta interfejs protokołu Profibus. Zdecydowano się jednak na zastosowanie zupełnie innego rozwiązania wykorzystującego interfejs sieci ELAN. Firma Siemens nie zaleca stosowania tego rozwiązania, ponieważ ELAN jest protokołem diagnostycznym, jednak można go w bardzo prosty sposób połączyć w sieć z komputerem klasy PC stosując jedynie prosty konwerter napięć, co drastycznie obniżyło koszty projektu. Zupełnie niespodziewanie udało się zaobserwować kilka wartych wspomnienia zjawisk.
