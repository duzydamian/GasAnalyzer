\section{Wstęp}
Pomiary są bardzo ważną czynnością w wielu współczesnych dziedzinach nauki i przemysłu. Niestety część z nich do dzisiaj jest wykonywana w sposób analogowy, zamiast w pełni korzystać z możliwości współczesnej aparatury pomiarowej. Podobnie było w przypadku projektu zrealizowanego przez autorów. Inspiracją na projekt są pomiary wykonywane przez pracowników Instytutu Maszyn i Urządzeń Energetycznych z Politechniki Śląskiej. Dysponują oni kilkoma analizatorami składu gazów firmy SIEMENS. Wszystkie urządzenia są standardowo wyposażone w interfejs ELAN oraz w~większości przypadków w~PROFIBUS. Autorzy zdecydowali się na wykorzystanie pierwszego z~nich ze względu na pełną dokumentację protokołu, otwartość implementacji oraz niskie koszty uruchomienia. Twórcy nie tylko przygotowali odpowiednie oprogramowanie, ale także uruchomili całe stanowisko pomiarowe.