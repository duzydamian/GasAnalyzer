\section{Aplikacja ,,GasAnalyzer''}
Oprogramowanie zostało stworzone w całości Javie. Dla ułatwienia kompilacji, zarządzanie zależnościami oraz wersjami zastosowano Apache Maven, które jest narzędziem automatyzującym budowę oprogramowania. Dzięki zastosowanym technologiom projekt można uruchomić na dowolnym komputerze wyposażonym w system operacyjny Windows, Linux lub Mac OXS Cocoa w wersjach 32 i 64 bitowych.

Projekt składa się z dwóch modułów co zostało pokazane na Rysunku~\ref{projectSchema}.Pierwszy z nich ELANNetwork odpowiada za odbieranie danych z sieci, ich weryfikację, przetwarzanie i przekazywanie do warstwy wyższej aplikacji.
Drugi moduł GasAnalyzerGUI jest graficznym interfejsem użytkownika (ang. Graphical User Interface, GUI). Odpowiada za przejrzystą prezentację danych użytkownikowi.

\tikzstyle{abstract}=[rectangle, draw=black, rounded corners, fill=blue!40, drop shadow,
        text centered, anchor=north, text=white, text width=4cm]
\tikzstyle{comment}=[rectangle, draw=black, rounded corners, fill=green, drop shadow,
        text centered, anchor=north, text=white, text width=4cm]
\tikzstyle{myarrow}=[->, >=open triangle 90, thick]
\tikzstyle{line}=[-, thick]

\begin{figure}[!htb] 	
\centering 	
\begin{tikzpicture}[node distance=1.1cm]
    \node (GasAnalyzer) [abstract, rectangle split, rectangle split parts=2]
        {
            \textbf{GasAnalyzer}
            \nodepart{second}wersja 0.1.0
        };
	\node (AuxNode01) [text width=4cm, below=of GasAnalyzer] {};
    \node (ELANNetwork) [abstract, rectangle split, rectangle split parts=2, left=of AuxNode01]
        {
            \textbf{ELANNetwork}
            \nodepart{second}wersja 0.1.0
        };
    \node (GasAnalyzerGUI) [abstract, rectangle split, rectangle split parts=2, right=of AuxNode01]
        {
            \textbf{GasAnalyzerGUI}
            \nodepart{second}wersja 0.1.0
        };    
    \draw[myarrow] (ELANNetwork.north) -- ++(0,0.1) -| (GasAnalyzer.south);
    \draw[line] (ELANNetwork.north) -- ++(0,0.1) -| (GasAnalyzerGUI.north); 
        
\end{tikzpicture}
\caption{Struktura projektu} 
\label{projectSchema}
 \end{figure} %

\begin{figure}[htbp]
 \centering
        \tikzstyle{background grid}=[draw, black!50,step=.25cm]
	\begin{tikzpicture}[node distance=2mm, auto]%, show background grid]
	\tikzset{
    	mynode/.style={rectangle,rounded corners,draw=black, top color=white, very thick, inner sep=4mm, 		text centered,font=\footnotesize},
	    myarrow/.style={ <->, shorten >=1pt, line width=1mm},
	    myarrow1/.style={ ->, line width=0.4mm},	    
   	    myarrow2/.style={ ->, shorten >=1mm, line width=0.4mm},
   	    myarrow3/.style={ ->, shorten >=4mm, line width=0.4mm},   	    
	    mylabel/.style={text centered, font=\scriptsize\bfseries} 
	} 
	\node[bottom color=cyan!40, mynode, text width=12cm] (gui) {Graficzny interfejs użytkownika};  
	\node[bottom color=cyan!40, mynode, text width=12cm, text height=2.5cm, below=1cm of gui] (elanNetwork) {Moduł ELAN Network}; 
	\node[bottom color=blue!40, mynode, text width=2cm, below=3cm of gui.173] (a) {Detekcja ramek}; 
	\node[bottom color=blue!40, mynode, text width=2cm, right=of a] (b) {Weryfikacja ramek}; 
	\node[bottom color=blue!40, mynode, text width=2cm, right=of b] (c) {Parsowanie ramek}; 	
	\node[bottom color=blue!40, mynode, text width=2cm, right=of c] (d) {Budowanie migawki}; 
	\node[bottom color=cyan!40, mynode, text width=12cm, below=1cm of elanNetwork] (rxtx) {Biblioteka RXTX}; 
	\node[bottom color=cyan!40, mynode, text width=12cm, below=1cm of rxtx] (conv) {Konwerter RS485 $\Leftrightarrow$ USB};   		
	\node[bottom color=cyan!40, mynode, text width=12cm, below=1cm of conv] (devices) {Urządzenia pomiarowe};	

	\draw[myarrow] (gui.south) -- (elanNetwork.north) node [midway,black] {Przetworzone dane};
	\draw[myarrow] (elanNetwork.south) -- (rxtx.north) node [midway,black] {Odczyt bajtów};	
	\draw[myarrow] (rxtx.south) -- (conv.north) node [midway,black] {USB};	
	\draw[myarrow] (conv.south) -- (devices.north) node [midway,black] {RS485};
	\draw[myarrow1] (a) -- (b);
	\draw[myarrow1] (b) -- (c);
	\draw[myarrow1] (c) -- (d);	
	\draw[myarrow2] (c.north) -- (elanNetwork.north) node [midway,black] {};
	\draw[myarrow3] (d.north) -- (elanNetwork.north) node [midway,black] {};
\end{tikzpicture} 
\caption{Struktura aplikacji.}
\label{appSchema}
\end{figure} %

\subsection{Możliwości}
Aplikacja oferuję mnóstwo przydatnych opcji:
\begin{enumerate}
\item Automatyczne wykrywanie urządzeń podpiętych do sieci,
\item Konfigurowalna precyzja wartości wyświetlanych w GUI i raportach,
\item Bieżący podgląd stanu sieci,
\item Bieżący podgląd stanu każdego urządzenia,
\item Zapis wartości zmierzonych ze wszystkich urządzeń z zadanym interwałem i opcjonalnym komentarzem.
\end{enumerate}

\subsection{Perspektywy}
\begin{enumerate}
\item Implementacja pozostałych możliwości protokołu,
\item Zdalne uruchomienie kalibracji urządzeń,
\item Zdalny odczyt błędów,
\item Rozbudowana detekcja urządzeń (rozpoznanie modelu),
\end{enumerate}