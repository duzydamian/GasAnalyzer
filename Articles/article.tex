%\documentclass[a4paper,12pt]{report}
\documentclass[a4paper,12pt]{article}
%\documentclass[a4paper,12pt]{book}
\usepackage{polski}
\usepackage[polish]{babel}
\usepackage[utf8]{inputenc}
\usepackage[top=2.5cm, bottom=2.5cm, left=3cm, right=2.5cm]{geometry}
\usepackage[justification=centering]{caption}
\usepackage{graphicx}
\usepackage{setspace}
\usepackage{ifthen}
\usepackage{a4wide}
\usepackage{fullpage}
\usepackage{verbatim}
\usepackage[usenames,dvipsnames]{color}
\usepackage{hyperref}
\usepackage{subfig}
\usepackage{listings}
\usepackage{mdwlist}
\usepackage{titlesec}
\usepackage{lipsum}
\usepackage{multirow}
\usepackage{enumitem}
\usepackage[clock]{ifsym}
\usepackage{clock}
\usepackage{amsmath}
\usepackage{cite}

\usepackage{pgfplots}
\usepackage{tikz}
\usetikzlibrary{automata,positioning,shapes,shadows,arrows,backgrounds,trees,fit,calc,decorations.pathreplacing,decorations.markings}

\hypersetup{
%	bookmarks=true,
	pdftitle={ELAN},
	pdfauthor={Damian Karbowiak, Grzegorz Powała},
	pdfsubject={ELAN},
	pdfkeywords={Politechnika Śląska ELAN},
	colorlinks=true,
	linkcolor=black,
	citecolor=black,
	urlcolor=black
	}
	
\let\subsubsubsection\paragraph
%\setcounter{secnumdepth}{6} % subsubparagraph ???
% that is, subsubsubsubsubsection :-)
\setcounter{secnumdepth}{4}

\newcommand{\tytul}{Projekt i realizacja stanowiska laboratoryjnego do badania zależności czasowych w sieci EtherCAT}
\newcommand{\data}{\today}
\newcommand{\promotor}{dr~inż. Jacek Stój}
\newcommand{\autor}{inż. Damian Karbowiak}
\newcommand{\konsultant}{}

\titlespacing{\section}{1cm}{*4}{*1.5}
\titlespacing{\subsection}{1cm}{*4}{*1.5}
\titlespacing{\subsubsection}{1cm}{*4}{*1.5}

\setitemize{itemsep=0pt, topsep=2pt}
\setenumerate{itemsep=0pt, topsep=2pt}

\linespread{1.3}

\begin{document}

\lstset{backgroundcolor=\color{white}, boxpos=c, captionpos=b}
\lstset{numbers=left, stepnumber=1, numbersep=10pt, frame=single}
\lstset{frameround=tttt}
\renewcommand{\lstlistlistingname}{\vspace*{-13mm}}
\renewcommand{\listfigurename}{\vspace*{-13mm}}
%\renewcommand*\l@figure[2]{\indent}
\renewcommand{\listtablename}{\vspace*{-13mm}}
\renewcommand*{\refname}{\vspace*{-13mm}}
\renewcommand{\lstlistingname}{Kod źródłowy} 

\newcommand{\rowstyle}[1]{\gdef\currentrowstyle{#1}%
#1\ignorespaces
}

%\begin{abstract}
\thispagestyle{plain}
\pagenumbering{Roman}
\noindent Damian KARBOWIAK, Grzegorz POWAŁA \\
Politechnika Śląska, Wydział Automatyki, Elektroniki i Informatyki, Instytut informatyki, Damian.Karbowiak@polsl.pl
Opiekun naukowy: dr inż Jacek STÓJ
Politechnika Śląska, Wydział Automatyki, Elektroniki i Informatyki, Instytut informatyki, jacek.stoj@polsl.pl
\begin{center}
\textbf{Zastosowanie protokołu ELAN w sieci pomiarowej}
\end{center}
\textbf{Streszczenie: }
Niniejszy artykuł opisuje wynik realizacji aplikacji do gromadzenia danych z analizatorów składu gazu firmy SIEMENS.
W artykule przedstawione zostało rozwiązanie zaproponowane i zaimplementowane przez autorów niniejszej publikacji. \\
\textbf{Słowa kluczowe: } ELAN, Siemens, pomiary, analizator gazów
\begin{center}
\textbf{Application of ELAN protocol in measurement network}
\end{center}
\textbf{Summary: }
This article describe result of realisation application to collect data from gas analyzers by SIEMENS company.\\
\textbf{Keywords: } ELAN, Siemens, measurement, gas analyzer
\clearpage
%\end{abstract}

\pagenumbering{arabic}
\tableofcontents
\addtocontents{toc}{\protect\vspace*{.05\baselineskip}}
\clearpage

\section{Wstęp}
%Część ta~zawiera wstępne informacje o~realizowanym projekcie. Zebrano w~nim wszystko to~co było dostępne zanim student przystąpił do~realizacji informatycznej części projektu. Opisano stanowisko laboratoryjne na~którym powstał projekt.

\subsection{Geneza}
Tematem projektu, którego dotyczy ta praca jest: „Syst". Pomysł na~projekt pojawił się po~zrealizowaniu przez autora projektu semestralnego z~przedmiotu Sterowniki PLC.

\subsection{Temat}
Głównymi celami pracy było napisanie oprogramowania

\subsection{Stanowisko}
\subsubsection{Stanowisko prototypowe}
\begin{figure}[!htb] 	\centering 	\includegraphics[width=0.8\textwidth]{images/schemat1} 	\caption{Schemat stanowiska prototypowego} \label{schemat1} \end{figure} 
Na potrzeby realizacji projektu stworzono stanowisko laboratoryjne, którego schemat przedstawia Rysunek~\ref{schemat1}. Składa się ono~z:
\begin{itemize}
\item Komputera,
\item ULTRAMAT-u 23
\item Konwertera ATC-850.
\end{itemize}
\indent
\indent Stanowisko to na potrzeby projektu rozbudowano o model magazynu wysokiego składowania. Poszczególne składowe stanowiska zostały bardziej szczegółowo opisane w~kolejnych podrozdziałach.

\subsubsection{Stanowisko docelowe}
\begin{figure}[!htb] 	\centering 	\includegraphics[width=0.8\textwidth]{images/schemat2} 	\caption{Schemat stanowiska docelowego} \label{schemat2} \end{figure} 

\subsubsection{Sterownik PLC}
Sterownik PLC wykorzystywany do realizacji projektu był wyposażony w następujące moduły:
\begin{enumerate}
\item SIMATIC S7-300, Jednostka centralna S7-300 CPU 315F-2 PN/DP,
\item SIMATIC S7-300, Zasilacz PS 307,
\item SIMATIC S7-300, Wejścia/Wyjścia cyfrowe SM 323,
\item SIMATIC S7-300, Wejścia/Wyjścia analogowe SM 334.
\end{enumerate}

\indent
\indent Sterownik podłączony jest do sieci lokalnej Ethernet w~laboratorium, więc komunikacja z~nim odbywa się tak samo jak z~każdym innym urządzeniem sieciowym. Podstawy programowania i korzystania ze sterowników autor poznał zapoznając się z odpowiednią literaturą \cite{plc1,plc2,plc4,plc5,plc6}.
Konfigurację sterownika wraz z modułami przedstawia Rysunek~\ref{conf}.

\subsubsection{Komputer}
Projekt w całości był realizowany na laptopie autora, podłączanym do~sieci w~laboratorium. Na~komputerze uruchomiane były dwie maszyny wirtualne. Na~jednej zainstalowane było środowisko Step~7 do~programowania sterownika, a~na~drugiej WinCC flexible 2008 do~tworzenia i~uruchamiania wizualizacji. Wizualizacje tworzone w środowisku WinCC flexible są~dedykowane do~paneli operatorskich, jednak ta~stworzona przez autora na~potrzeby projektu była uruchamiana na komputerze za~pomocą runtime system.

\subsection{Analiza tematu}
Analiza tematu polegała przede wszystkim na zapoznaniu się z~narzędziami programistycznymi do~tworzenia oprogramowania sterownika oraz wizualizacji.
W~wyniku analizy autor poznał podstawy języków: LAD~\cite{step1,step2,step3}, STL~\cite{step1,step2,step3}, FBD~\cite{step1,step2,step3}, GRAPH~\cite{step3}, SCL~\cite{scl1,scl2,scl3} i AWL do~tworzenia programu sterownika oraz VBScript do~tworzenia skryptów w~wizualizacji. Poznanie tych podstaw pozwoliło dobrać język odpowiedni do~realizacji poszczególnych zadań.

\subsection{Założenia}
%Stworzone oprogramowanie dla Robota Fishertechnik ma działać na sterownikach firmy Siemens oraz ma zostać stworzone przy użyciu środowiska Step 7. Funkcjonalności robota, jakie mają wchodzić w~skład projektu, to:
Oprogramowanie dla Robota Fishertechnik powinno zostać stworzone przy użyciu środowiska Step 7 oraz działać na sterownikach firmy Siemens. Funkcjonalności robota wchodzące w~skład projektu, to:
\begin{itemize}
\item sterowanie ręczne z~pilota podłączonego bezpośrednio do~sterownika,
\item sterowanie ręczne z~wizualizacji,
\item sterowanie automatyczne, 
\item wizualizacja stanu magazynu,
\item umożliwienie korzystania z~magazynu zarówno poprzez sterowanie ręczne, jak i~przy użyciu zautomatyzowanych poleceń dostępnych z~poziomu wizualizacji.
\end{itemize}
\indent
\indent Powyżej zostały wymienione założenia podstawowe, jednak autor nie wyklucza zrealizowania dodatkowych zadań, które nie zostały zamieszczone w~pierwotnej koncepcji realizacji projektu.
%Zamieszczone tu założenia są podstawowe, ale niewykluczone jest zrealizowanie przez autora dodatkowych zadań nie zaplanowanych przed rozpoczęciem realizacji projektu.

\subsection{Plan pracy}
Realizacja projektu została podzielona na następujące etapy:
\begin{itemize}
\item Przygotowanie stanowiska, zebranie odpowiednich materiałów i~literatury,
\item Analiza wymagań funkcjonalnych aplikacji,
\item Projektowanie struktury oprogramowania i~interfejsów wymiany danych,
\item Implementacja,
\item Testowanie i~uruchamianie,
\item Przedstawienie projektu i~ewentualne korekty.
\end{itemize}
\indent
\indent Powyższy plan pracy stanowił dla autora wyznacznik kolejnych działań. Jednak powszechnie wiadomo, że w~praktyce poszczególne punkty są~wymienne i~wpływają na siebie wzajemnie.
 %
\section{Opis protokołu ELAN}

ELAN (ang. Economical Local Area Network), czyli ekonomiczna sieć lokalna został wprowadzony przez firmę SIEMENS w swoich analizatorach składu gazu. Protokół ten według twórców został wprowadzony jako ekonomiczny interfejs szeregowy do transmisji wartości mierzonych pomiędzy analizatorami oraz prostej komunikacji z komputerami PC dla celów testowych i serwisowych.
Zostało wprowadzone ograniczenie maksymalnej liczby urządzeń pracujących w sieci do 14 (2 urządzenia kontrolne/komputery oraz do 12 analizatorów). Obsługiwane analizatory firmy SIEMENS:
\begin{itemize}
\item ULTRAMAT 6
\item OXYMAT 6 / OXYMAT 61
\item CALOMAT 6
\item ULTRAMAT 23
\end{itemize}

\begin{table}[h]
\centering
\begin{tabular}{|l|l|}
\hline Poziom & RS485 \\ 
\hline Szybkość transmisji & 9600 \\ 
\hline Bity danych & 8 \\ 
\hline Bit startu & 1 \\ 
\hline Bit stopu & 1 \\ 
\hline Kontrola parzystości & nie \\ 
\hline no ECHO &  \\ 
\hline 
\end{tabular} 
\caption{Parametry interfejsu}
\label{tab:parametry}
\end{table}

\subsection{Analiza statystyczna protokołu}
TCS -- Rozmiar transmitowanego znaku (ang. transmitted char size)\\
CS -- rozmiar znaku (ang. char size)\\
SpB -- bit stopu (ang. Stop bit) \\
StB -- bit startu (ang. Start bit)
$$TCS = CS + StB + SpB= 8+1+1=10 [b]$$
CPS -- liczba znaków na sekundę (ang. chars per second) \\
CBR -- znakowa szybkość transmisji (ang. char baud rate) \\
BR -- szybkość transmisji (ang. baud rate) \\
Szybkość transmisji wg dokumentacji: 9600 bps (bitów na sekundę)
$$CBR=BR/TCS=9600/10=960 [cps]$$

\subsection{Ramka}
Każda ramka w sieci  wygląda ogólnie jak na Rysunku~\ref{elan:ramka}
\begin{figure}[htbp]
 \centering
        \tikzstyle{background grid}=[draw, black!50,step=.25cm]
	\begin{tikzpicture}[node distance=1.5mm, auto]%, show background grid]
	\tikzset{
    	mynode/.style={rectangle,rounded corners,draw=black, top color=white, very thick, inner sep=4mm, 		text centered,font=\footnotesize},
    	mynodemini/.style={rectangle,rounded corners,draw=black, top color=white, thick, inner sep=2mm, text centered,font=\scriptsize},    	
	    myarrow/.style={->, >=latex', shorten >=1pt, ultra thick},
	    myline/.style={-, =latex', shorten >=1pt, rounded corners, ultra thick},
	    mylabel/.style={text centered, font=\scriptsize\bfseries} 
	} 
	\node[bottom color=gray!50, mynode, text width=2.3cm] (start) {Start};  
	\node[bottom color=gray!50, mynodemini, below=of start.213, text width=1cm] (dle) {DLE (10H)};	
	\node[bottom color=gray!50, mynodemini, right=of dle, text width=1cm] (soh) {SOH (01H)};  
	
	\node[bottom color=yellow!50, mynode, right=of start, text width=6.3cm] (used_data) {USED DATA};
	\node[bottom color=yellow!50, mynodemini, below=of used_data.190] (ud1) {TA};  	 
	\node[bottom color=yellow!50, mynodemini, right=of ud1] (ud2) {SA};
	\node[bottom color=yellow!50, mynodemini, right=of ud2] (ud3) {CCS};
	\node[bottom color=yellow!50, mynodemini, right=of ud3] (ud4) {CS};
	\node[bottom color=yellow!50, mynodemini, right=of ud4] (ud5) {Komenda};
	\node[bottom color=yellow!50, mynodemini, right=of ud5] (ud6) {Dane};			

	\node [fit=(ud1) (ud2) (ud3) (ud4) (ud5) (ud6)] (fit) {};  
 	%\draw [decorate, xshift=-20pt,line width=4pt] (fit.south east) -- (fit.north east);
	\draw [decorate,decoration={brace,amplitude=10pt}, line width=1pt] (fit.south east) -- (fit.south west);		
	\node[mylabel, below=4mm of fit] (fits) {max. 68 characters, 10H zawsze podwojone};
			
	\node[bottom color=gray!50, mynode, right=of used_data, text width=2.2cm] (eot) {End of \\ transmission};
	\node[bottom color=gray!50, mynodemini, below=of eot.222, text width=1cm] (dle2) {DLE (10H)};
	\node[bottom color=gray!50, mynodemini, right=of dle2, text width=1cm] (etx) {ETX (03H)}; 	 
	
	\node[bottom color=gray!50, mynode, right=of eot] (crc16) {CRC16};
	
	\node[mylabel, below=2cm of soh] (dal) {TA -- Target Address};
	\node[mylabel, right=of dal] (sal) {SA -- Source Address};
	\node[mylabel, right=of sal] (padl) {Pad. -- Payload};
	\node[mylabel, right=of padl, text width=4cm] (fcsl) {FCS -- Frame Check Sequance (CRC)};			
	
\end{tikzpicture} 
\caption{Ramka w protokole ELAN.}
\label{elan:ramka}
\end{figure} %

\subsection{Topologia}
Jak już zostało to opisane wcześniej w sieci może pracować do 12 analizatorów oraz komputer co zostało predstawione na Rysunku~\ref{elan:topologia}.
\input{tikz/topology} % %
\section{Aplikacja ,,GasAnalyzer''}
Oprogramowanie zostało stworzone w całości Javie. Dla ułatwienia kompilacji, zarządzanie zależnościami oraz wersjami zastosowano Apache Maven, które jest narzędziem automatyzującym budowę oprogramowania. Dzięki zastosowanym technologiom projekt można uruchomić na dowolnym komputerze wyposażonym w system operacyjny Windows, Linux lub Mac OXS Cocoa w wersjach 32 i 64 bitowych.

Projekt składa się z dwóch modułów co zostało pokazane na Rysunku~\ref{projectSchema}.Pierwszy z nich ELANNetwork odpowiada za odbieranie danych z sieci, ich weryfikację, przetwarzanie i przekazywanie do warstwy wyższej aplikacji.
Drugi moduł GasAnalyzerGUI jest graficznym interfejsem użytkownika (ang. Graphical User Interface, GUI). Odpowiada za przejrzystą prezentację danych użytkownikowi.

\tikzstyle{abstract}=[rectangle, draw=black, rounded corners, fill=blue!40, drop shadow,
        text centered, anchor=north, text=white, text width=4cm]
\tikzstyle{comment}=[rectangle, draw=black, rounded corners, fill=green, drop shadow,
        text centered, anchor=north, text=white, text width=4cm]
\tikzstyle{myarrow}=[->, >=open triangle 90, thick]
\tikzstyle{line}=[-, thick]

\begin{figure}[!htb] 	
\centering 	
\begin{tikzpicture}[node distance=1.1cm]
    \node (GasAnalyzer) [abstract, rectangle split, rectangle split parts=2]
        {
            \textbf{GasAnalyzer}
            \nodepart{second}wersja 0.1.0
        };
	\node (AuxNode01) [text width=4cm, below=of GasAnalyzer] {};
    \node (ELANNetwork) [abstract, rectangle split, rectangle split parts=2, left=of AuxNode01]
        {
            \textbf{ELANNetwork}
            \nodepart{second}wersja 0.1.0
        };
    \node (GasAnalyzerGUI) [abstract, rectangle split, rectangle split parts=2, right=of AuxNode01]
        {
            \textbf{GasAnalyzerGUI}
            \nodepart{second}wersja 0.1.0
        };    
    \draw[myarrow] (ELANNetwork.north) -- ++(0,0.1) -| (GasAnalyzer.south);
    \draw[line] (ELANNetwork.north) -- ++(0,0.1) -| (GasAnalyzerGUI.north); 
        
\end{tikzpicture}
\caption{Struktura projektu} 
\label{projectSchema}
 \end{figure} %

\subsection{Możliwości}
Aplikacja oferuję mnóstwo przydatnych opcji:
\begin{enumerate}
\item Automatyczne wykrywanie urządzeń podpiętych do sieci,
\item Konfigurowalna precyzja wartości wyświetlanych w GUI i raportach,
\item Bieżący podgląd stanu sieci,
\item Bieżący podgląd stanu każdego urządzenia,
\item Zapis wartości zmierzonych ze wszystkich urządzeń z zadanym interwałem i opcjonalnym komentarzem.
\end{enumerate}

\subsection{Perspektywy}
\begin{enumerate}
\item Implementacja pozostałych możliwości protokołu,
\item Zdalne uruchomienie kalibracji urządzeń,
\item Zdalny odczyt błędów,
\item Rozbudowana detekcja urządzeń (rozpoznanie modelu),
\end{enumerate} %
\section{Podsumowanie}
Stworzona aplikacja oparta, o zaproponowane rozwiązanie, pozwoliła znacząco uprościć i przyśpieszyć proces pomiarowy.Rozwiązanie to jest bardzo tanie i proste w uruchomieniu. Stworzone oprogramowanie pozwala na swobodne modyfikowanie listy urządzeń, które są automatycznie wykrywane. \vspace{1cm} \\
EWENTUALNIE!!!!!!! z naszego raportu \\
Nadrzędnym celem projektu było stworzenie oprogramowania gromadzącego i zapisującego dane z analizatorów spalin firmy Siemens. Urządzenia te są wyposażone w dwa interfejsy komunikacyjne, w tym sugerowany przez producenta interfejs protokołu Profibus. Zdecydowano się jednak na zastosowanie zupełnie innego rozwiązania wykorzystującego interfejs sieci ELAN. Firma Siemens nie zaleca stosowania tego rozwiązania, ponieważ ELAN jest protokołem diagnostycznym, jednak można go w bardzo prosty sposób połączyć w sieć z komputerem klasy PC stosując jedynie prosty konwerter napięć, co drastycznie obniżyło koszty projektu. Zupełnie niespodziewanie udało się zaobserwować kilka wartych wspomnienia zjawisk.
 %
\section{Bibliografia}
Literatura, która została wykorzystana przez autorów w czasie powstawania projektu, którą opisuje niniejsza dokumentacja.

\begin{thebibliography}{9}
%\begin{enumerate}
%\item 
\bibitem{plc1} 
Jerzy Kasprzyk: 
\emph{"Programowanie sterowników przemysłowych"},
Wydawnictwa Naukowo-Techniczne WNT, 
Warszawa, 
2007      

\bibitem{elan} 
Dokumentacja producenta: 
\emph{„ELAN Interface Description”}, 
sierpień 2006

\bibitem{kurs1} 
Materiały szkoleniowe:
„SIMATIC S7 - Kurs podstawowy”

\end{thebibliography} %

\section{Spis rysunków, tablic i kodów źródłowych}
\subsection{Spis rysunków}
\newlength{\fig}
\settowidth{\fig}{Rysunek\,99:~}
\renewcommand*\numberline[1]{\llap{\makebox[\fig][l]{Rysunek\,#1:~}}}
\makeatletter
\renewcommand*\l@figure[2]{\leftskip\fig\noindent#1\par}
\renewcommand*\l@figure{\@dottedtocline{1}{3cm}{0cm}}
\makeatother
\listoffigures

\subsection{Spis tablic}
%\renewcommand*\numberline[1]{Tablica\,#1: \indent}
\settowidth{\fig}{Tablica\,99:~}
\renewcommand*\numberline[1]{\llap{\makebox[\fig][l]{Tablica\,#1:~}}}
\makeatletter
\renewcommand*\l@table[2]{\leftskip\fig\noindent#1\par}
\renewcommand*\l@table{\@dottedtocline{1}{2.8cm}{0cm}}
\makeatother
\listoftables
\subsection{Spis kodów źródłowych}
\settowidth{\fig}{Kod źródłowy,99:~}
\renewcommand*\numberline[1]{\llap{\makebox[\fig][l]{Kod źródłowy\,#1:~}}}
%\renewcommand*\numberline[1]{Kod źródłowy\,#1:\indent}
\makeatletter
\renewcommand*\l@lstlisting[2]{\leftskip\fig\noindent#1\par}
\renewcommand*\l@lstlisting{\@dottedtocline{1}{4.1cm}{0cm}}
\makeatother
\lstlistoflistings

\end{document}
