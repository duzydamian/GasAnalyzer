\section{Aplikacja ,,GasAnalyzer''}
Oprogramowanie zostało stworzone w całości Javie. Dla ułatwienia kompilacji, zarządzanie zależnościami oraz wersjami zastosowano Apache Maven, które jest narzędziem automatyzującym budowę oprogramowania. Dzięki zastosowanym technologiom projekt można uruchomić na dowolnym komputerze wyposażonym w system operacyjny Windows, Linux lub Mac OXS Cocoa w wersjach 32 i 64 bitowych.

Projekt składa się z dwóch modułów co zostało pokazane na Rysunku~\ref{projectSchema}.Pierwszy z nich ELANNetwork odpowiada za odbieranie danych z sieci, ich weryfikację, przetwarzanie i przekazywanie do warstwy wyższej aplikacji.
Drugi moduł GasAnalyzerGUI jest graficznym interfejsem użytkownika (ang. Graphical User Interface, GUI). Odpowiada za przejrzystą prezentację danych użytkownikowi.

\tikzstyle{abstract}=[rectangle, draw=black, rounded corners, fill=blue!40, drop shadow,
        text centered, anchor=north, text=white, text width=4cm]
\tikzstyle{comment}=[rectangle, draw=black, rounded corners, fill=green, drop shadow,
        text centered, anchor=north, text=white, text width=4cm]
\tikzstyle{myarrow}=[->, >=open triangle 90, thick]
\tikzstyle{line}=[-, thick]

\begin{figure}[!htb] 	
\centering 	
\begin{tikzpicture}[node distance=1.1cm]
    \node (GasAnalyzer) [abstract, rectangle split, rectangle split parts=2]
        {
            \textbf{GasAnalyzer}
            \nodepart{second}wersja 0.1.0
        };
	\node (AuxNode01) [text width=4cm, below=of GasAnalyzer] {};
    \node (ELANNetwork) [abstract, rectangle split, rectangle split parts=2, left=of AuxNode01]
        {
            \textbf{ELANNetwork}
            \nodepart{second}wersja 0.1.0
        };
    \node (GasAnalyzerGUI) [abstract, rectangle split, rectangle split parts=2, right=of AuxNode01]
        {
            \textbf{GasAnalyzerGUI}
            \nodepart{second}wersja 0.1.0
        };    
    \draw[myarrow] (ELANNetwork.north) -- ++(0,0.1) -| (GasAnalyzer.south);
    \draw[line] (ELANNetwork.north) -- ++(0,0.1) -| (GasAnalyzerGUI.north); 
        
\end{tikzpicture}
\caption{Struktura projektu} 
\label{projectSchema}
 \end{figure} %

\subsection{Możliwości}
Aplikacja oferuję mnóstwo przydatnych opcji:
\begin{enumerate}
\item Automatyczne wykrywanie urządzeń podpiętych do sieci,
\item Konfigurowalna precyzja wartości wyświetlanych w GUI i raportach,
\item Bieżący podgląd stanu sieci,
\item Bieżący podgląd stanu każdego urządzenia,
\item Zapis wartości zmierzonych ze wszystkich urządzeń z zadanym interwałem i opcjonalnym komentarzem.
\end{enumerate}

\subsection{Perspektywy}
\begin{enumerate}
\item Implementacja pozostałych możliwości protokołu,
\item Zdalne uruchomienie kalibracji urządzeń,
\item Zdalny odczyt błędów,
\item Rozbudowana detekcja urządzeń (rozpoznanie modelu),
\end{enumerate}